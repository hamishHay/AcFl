\section{Conclusions}

This work demonstrates the successful implementation of the full Melosh model of acoustic fluidization into the iSALE hydrocode. The Melosh model results presented here agree with the predicted evolution of acoustic energy in both space and time \citep{melosh1979acoustic}. A series of results that explore a small region of the Melosh model parameter space are also presented and compared to an alternative, widely used approximation of acoustic fluidization - the block oscillation model. Certain values of these parameters are shown to produce predicted crater morphologies for a range of terrestrial impacts, from simple craters to peak ring complex craters. However, Melosh model crater depths deviate more strongly from terrestrial scaling laws for complex craters than do those of the block model.

Crater collapse in the Melosh model is facilitated by the shallow regeneration of localised acoustic energy, while collapse in the block model is the result of long lived vibrations extending deep into the target. Localised, shallow acoustic vibrations in the Melosh model result in different styles of subcrater deformation between the two models. In particular, the two acoustic fluidization models predict different modes of peak ring formation that correspond to alternative hypothesis based on observational evidence. To further explore the differences between the two models, an extensive exploration of Melosh model parameter space at high resolution is required, as well as a detailed comparison of subcrater deformation between the two models and observational evidence. 
\section{Methodology}

\subsection{Energy Balance Solution}

In order to first implement the acoustic energy evolution equation (\ref{eq:dedt}) into iSALE, the acoustic energy density was converted to an energy per unit mass, using $\mathcal{E}_{v}=E/\rho$ \citep{collins2003acoustic}:

\begin{equation}\label{eqdedtmass}
\frac{d \mathcal{E}_{v}}{d t} = \frac{\xi}{4} \nabla^{2}\mathcal{E}_{v} - \frac{c_{p}}{\lambda Q}\mathcal{E}_{v} + e \frac{\tau \dot{\epsilon}}{2\rho}.\vspace{0.2cm}
\end{equation}

This is necessary, as iSALE stores specific energy rather than energy density.

To satisfy conservation, energy available for regeneration of acoustic vibrations must be removed from the internal energy of the rock material. In iSALE, the rate of change of internal energy per unit mass, $I$, is given as \citep{collins2002numerical}:

\begin{equation}\label{eq:iSaleInternalEnergy}
\frac{d \left( \rho I\right)}{d t} =\frac{d \left(p + q\right)}{d t} + \tau \dot{\epsilon},\vspace{0.2cm}
\end{equation}

where the terms on the right hand side represents the work done against compression and deformation, respectively. To account for energy transfer, both the acoustic energy lost in heat dissipation and the amount of distortional energy available for acoustic regeneration must be included in Equation \ref{eq:iSaleInternalEnergy} \citep{collins2002numerical}:

\begin{equation}\label{eq:iSaleInternalEnergyAcoustic}
\frac{d \left(\rho I\right)}{d t} =\frac{d \left(p + q\right)}{d t} + \tau \dot{\epsilon}\left(1-\frac{e}{2}\right)+\frac{c_{p}}{\lambda Q}\mathcal{E}_{v}.\vspace{0.2cm}
\end{equation}

Finally, the first term from equation \ref{eqdedtmass} must also be included to account for the scattering of acoustic energy \citep{collins2002numerical}. This is simply diffusion of acoustic energy, as governed by the Laplace equation, 

\begin{equation} \label{eq:cyldiff}
\frac{d \mathcal{E}_{v} }{d t} = \frac{\xi}{4}  \left( \frac{1}{r} \frac{d}{dr} \left( r \frac{d \mathcal{E}_{v}}{dr} \right) + \frac{d^{2} \mathcal{E}_{v}}{dz^{2}}   \right) .\vspace{0.2cm}
\end{equation} 

in a cylindrical coordinate system with angular symmetry about the z-axis.

Equations \ref{eqdedtmass}, \ref{eq:iSaleInternalEnergyAcoustic}, and \ref{eq:cyldiff}, as well as the acoustic fluidization strength model (Equation \ref{eq:yield_vib}) were successfully implemented into the iSALE hydrocode. This is further detailed in \ref{app:modifications}.

\subsection{Models and their Parameters}\label{sec:parameters}

To test the implementation of the Melosh model and contrast its behaviour against the block model, three different scenarios of crater formation were used. Both acoustic fluidization models were applied to each scenario on separate simulations. Table \ref{tb:ABC_table} displays all the acoustic fluidization parameters used in each scenario, as well as the model setup parameters. Parameters shared throughout all scenarios are presented in Table \ref{tb:static}. The static strength of rock in each case was assumed to be the same.

\begin{table}[!t]
\small
\centering
\begin{tabular*}{\linewidth}{@{\extracolsep{\fill} }p{3cm} c c c}
\toprule
& \multicolumn{3}{c}{Scenario} \\ \cmidrule{2-4}
Parameter & A & B & C \\ \midrule
Impactor radius, $a$ (km) & 7.200 & 0.720 & 0.072 \\
Impactor resolution & 36 & 24 & 18 \\ 
Grid spacing (m) & 200 & 30 & 4 \\  
Scattering diffusivity, $\xi$ (m$^{2}$ s$^{-1}$) & 10$^3$ & 10$^3$ & 10$^3$ \\
Dissipation quality factor, $Q$ & 10-400 & 10-400 & 10-400 \\
Vibrational wavelength, $\lambda$ (m) & 1440-360 & 144-36 & 14.4-3.6 \\
Regeneration parameter, $e$ & 0.1 & 0.1 & 0.1 \\ \bottomrule
\end{tabular*}
\caption{Melosh model acoustic fluidisation parameters used in each crater formation scenario. Global parameters are given in Table \ref{tb:static}}\label{tb:ABC_table}
\end{table}

\begin{table*}[!bh]
\small
\centering

\begin{tabular*}{\linewidth}{@{\extracolsep{\fill} }p{2cm} l l}
\toprule
Parameter & Description & Value \\ \midrule
$g$ & acceleration of gravity & 9.81 m s$^{-2}$\\
$U$ & impactor velocity & 1.2 $\times$ 10$^{4}$ m s$^{-1}$\\
$T_{\text{melt}}$ & melting temperature & 1673 K \\  
$\mu_{\text{int}}$ & intact material internal coefficient of friction & 2.0 \\
$\mu_{\text{dam}}$ & damaged material internal coefficient of friction  & 0.6\\
$Y_{\text{int}}$& intact material yield strength  & 10$^7$ kg m$^{-1}$ s$^{-2}$\\
$Y_{\text{dam}}$& damaged material yield strength & 10$^4$ kg m$^{-1}$ s$^{-2}$\\
$mat_{\text{t}}$& target material type & granite\\
$mat_{\text{p}}$& impactor material type & granite\\
$\rho_{\text{t}}$ & target material density & 2630 kg m$^{-3}$\\ 
$\rho_{\text{p}}$ & impactor material density & 2630 kg m$^{-3}$ \\  \bottomrule
\end{tabular*}
\caption{Static model parameters used in all Melosh and block model scenarios.}\label{tb:static}
\end{table*}

The square of the compressional wave velocity divided by the shear wave velocity, $\psi$, was maintained at 10 throughout all simulations, following the reasoning provided by \citet{collins2003acoustic}. The single values for $e$ and $\xi$ used here were selected from the range of possible values demonstrated by \citet{collins2003acoustic} in their work simulating the mobility of large rock avalanches. 

The two main parameters that were varied, as shown in Table \ref{tb:ABC_table}, were the vibrational wavelength, $\lambda$, and the quality dissipation factor, $Q$. Both of these are unconstrained. $Q$ is the most constrained as it is similar to the seismic $Q$ \citep{melosh1996dynamical}. A typical value of the seismic $Q$ in the upper crust is 500 \citep{anderson1989theory}, which accounts for both elastic energy converted to thermal energy and lost through scattering. In contrast, the full acoustic fluidization $Q$ only includes conversion to thermal energy, meaning 500 is likely an underestimate. \citet{melosh1999impact}, demonstrated that $Q$ must be of the order 10-100 for block model results to realistically fit observed crater profiles. As such, $Q$ was varied between 10-400 in this study. 

The only constraint set on the vibrational wavelength, $\lambda$, is that it must be less than the final crater diameter, but much greater than the material grain size \citep{melosh1979acoustic,collins2002numerical}. Following this condition $\lambda$ was assumed to scale proportionally with impactor size \citep{collins2003acoustic}. A similar assumption is made in the block model, where the block size is also proportional to impactor size \citep{ivanov1997block}. This assumption is key: if $\lambda$ scales proportionally with decreasing impactor size, the dissipation term in equation \ref{eqdedtmass} will become large, and acoustic energy will dissipate more rapidly for smaller impacts. Consequently, $\lambda$ was maintained as a fraction of the impactor radius, ranging from $0.2a$ to $0.05a$.

Block model simulations used parameters similar to recent work \citetext{e.g., \citet{wunnemann2003numerical}}. 
